%%%%%%%%%%%%%%%%%%%%%%%%%%%%%%%%%%%%%%%%%%%%%%%%%%%%%%%%%%%%%%%%%%%%%%%%%%
\documentclass[twocolumns,10pt,a4j]{jarticle}
%%%%%%%%%%%%%%%%%%%%%%%%%%%%%%%%%%%%%%%%%%%%%%%%%%%%%%%%%%%%%%%%%%%%%%%%%%
\usepackage{amsmath}
\usepackage[dvipdfmx]{graphicx}% Include figure files
%\usepackage{txfonts}
\usepackage{setspace}
\usepackage{bm}% bold math
\usepackage{here}
\usepackage{array}
\usepackage[T1]{fontenc}
\usepackage{etoolbox}
\usepackage[top=10truemm,bottom=25truemm,left=25truemm,right=20truemm]{geometry}%余白
\usepackage{layout}
\usepackage{wrapfig}
\renewcommand{\abstractname}{} 
\renewcommand{\figurename}{\small{Fig.}}
\renewcommand{\thefootnote}{\fnsymbol{footnote}}
\renewcommand{\baselinestretch}{0.80}
\renewcommand{\thesection}{\arabic{section}.}
\usepackage{indentfirst}
\usepackage{otf}
%\usepackage{multicol}
\pagestyle{empty}
%%%%%%%%%%%%%%%%%%%%%%%%%%%%%%%%%%%%%%%%%%%%%%%%%%%%%%%%%%%%%%%%%%%%%%%%%%
\makeatletter%% プリアンブルで定義する場合は必須
\patchcmd{\maketitle}{\@fnsymbol}{\@alph}{}{}  % Footnote numbers from symbols to small letters
\long\def\@makecaption#1#2{% \@makecaption を再定義します
  \vskip\abovecaptionskip
  \iftdir\sbox\@tempboxa{#1\hskip1zw#2}%
  \else\sbox\@tempboxa{#1~ #2}% ここの : を ~ に変更する
  \fi
  \ifdim \wd\@tempboxa >\hsize% 
  \iftdir #1\hskip1zw#2\relax\par
  \else #1~ #2\relax\par\fi% ここの : を ~ に変更する
  \else
  \global \@minipagefalse
  \hbox to\hsize{\hfil\box\@tempboxa\hfil}% センタリング
  %   \hbox to\hsize{\box\@tempboxa\hfil}%      左詰め
  %   \hbox to\hsize{\hfil\box\@tempboxa}%      右詰め
  \fi
  \vskip\belowcaptionskip}

  \DeclareRobustCommand\cite{\unskip
  \@ifnextchar[{\@tempswatrue\@citex}{\@tempswafalse\@citex[]}}
  \def\@cite#1#2{$^{[\hbox{\scriptsize{#1\if@tempswa ,#2\fi}]}}$}
  \def\@biblabel#1{[#1]}
  \makeatother%% プリアンブルで定義する場合は必須
  \setlength{\columnsep}{8  truemm}
  \setlength{\linewidth}{81 truemm}
  %%%%%%%%%%%%%%%%%%%%%%%%%%%%%%%%%%%%%%%%%%%%%%%%%%%%%%%%%%%%%%%%%%%%%%
  \title{\Large せん断流下におけるマイクロスイマー希薄分散系の動的挙動\vspace{-3truemm}}
  \author{\large 化学プロセス工学コース 移動現象論分野 荊尾太雅\vspace{-10zh}}
  \date{}
  %%%%%%%%%%%%%%%%%%%%%%%%%%%%%%%%%%%%%%%%%%%%%%%%%%%%%%%%%%%%%%%%%%%%%%
  \begin{document}

  %% start abstract

  \twocolumn[
    \maketitle\thispagestyle{empty}
    \vspace{-10truemm}
    \begin{quote}
      \normalsize
マイクロスイマーとは,水中の微生物に代表される粘性流体中を泳動により自己推進する微小な物体の総称である.
マイクロスイマーが分散した流体は,自己推進しないコロイド粒子などが分散した流体とは,大きく性質が異なることが知られている.
本研究では,球状のモデルマイクロスイマーである Squirmer 粒子に bottom heavy 性を付加し,
直接数値計算によりSquirmer 粒子希薄分散系のダイナミクスとレオロジー特性について調べた.
    \vspace{3truemm}
    \end{quote}
  ]

  %% end abstract
  %% start 1.緒言

  \noindent
  \textbf{\large 1.緒言}
  \par
マイクロスイマー単体やその集団の動的挙動を理解することは,バイオフィルムの形成過程の説明やドラッグデリバリーシステムへの応用など,工学的な用途に有用である.
本研究では,Squirmerモデルを用いた直接数値計算を行い,bottom heavy 性の付加によるマイクロスイマー単体の動的挙動を詳細に検討する.

  %% end 1.
  %% start 2.計算手法

  \vspace{1.0truemm}
  \noindent
  \textbf{\large 2.計算手法}
  \par
Squirmerモデルでは,球形の粒子表面において粒子と流体の速度差が式\eqref{boundary_I}で表される様なsliding境界条件を用いる\cite{1}.

  \vspace{-3truemm}
    \begin{equation}
      \boldsymbol{u}^s = B_1 \left( \sin{\theta} + \frac{\alpha}{2} \sin{2\theta} \right) \hat{\boldsymbol{\theta}}
      \label{boundary_I}
    \end{equation}
  \vspace{-4truemm}

  \noindent
ここで,$\boldsymbol{u}^s$はスイマー表面の重心に対するslide速度,
$B_1$は球面slide速度を多項式展開した際の係数第1項,
$\theta$は動径ベクトルと推進方向との間の極角,
$\hat{\boldsymbol{\theta}}$は単位極角ベクトル,
$\alpha$はスイマーの種類を表す定数である.
$\alpha>0$をPuller型,
$\alpha=0$をNeutral型,
$\alpha<0$をPusher型と呼ぶ.
直接数値計算は,Squirmerモデルに対してSmoothe Profile法 (SPM)を用いて行った\cite{2}.
流体の支配方程式として非圧縮性流体におけるNavier-Stokes方程式\eqref{Navier_Stokes}を用いた.
また,スイマーの時間発展はNewton-Euler方程式\eqref{Newton_Euler}で与えた.

\par
1.Navier-Stokes方程式\\
  \vspace{-3truemm}
  \small
  \begin{equation}
    \begin{split}
      \boldsymbol{\nabla}\cdot\boldsymbol{u} &= 0 \\
      \rho_\mathrm{f} \left(\partial_\mathrm{t} + \boldsymbol{u} \cdot \boldsymbol{\nabla} \right) \boldsymbol{u} &= \boldsymbol{\nabla} \cdot \boldsymbol{\sigma} + \rho_\mathrm{f} \left( \phi \boldsymbol{f}_\mathrm{p} + \boldsymbol{f}_\mathrm{sq} + \boldsymbol{f}_\mathrm{s} \right)
    \end{split}
    \label{Navier_Stokes}
  \end{equation}
  \normalsize
  \vspace{-4truemm}

  \noindent
ここで,$t$は時間,$\boldsymbol{u}$は全速度場,$\rho_\mathrm{f}$は流体の質量密度,$\boldsymbol{\sigma}$は流体の応力テンソル,
$\phi \boldsymbol{f}_\mathrm{p}$は粒子の剛直性を保証する体積力,
$\boldsymbol{f}_\mathrm{sq}$はスイマーの泳動による力,
$\boldsymbol{f}_\mathrm{s}$は式\eqref{zigzag_flow}のように表される速度を流体に生じさせる外力である.
  \vspace{-3truemm}
  \small
  \begin{equation}
    v_x(y) =
    \begin{cases}
      \dot{\gamma} (-y - L_y/2) & (-L_y/2 < y \le -L_y/4) \\
      \dot{\gamma} y            & (-L_y/4 < y \le L_y/4) \\
      \dot{\gamma} (-y + L_y/2) & (L_y/4 < y \le L_y/2)\\
    \end{cases}
    \label{zigzag_flow}
  \end{equation}
  \normalsize
  \vspace{-4truemm}

  \noindent
 ここで,$\dot{\gamma}$はせん断速度,$y$は系のy座標,$L_y$は系のy軸方向の大きさである.

  \par
2.Newton-Euler方程式\\
  \vspace{-4truemm}
  \small
  \begin{equation}
    \begin{split}
      \dot{\boldsymbol{R}}_i = \boldsymbol{V}_i &, \quad \boldsymbol{\dot{Q_i}} = \mathrm{skew} (\boldsymbol{\Omega_i}) \cdot \boldsymbol{Q_i} \\
      M_p \dot{\boldsymbol{V}_i} = \boldsymbol{F}_i^\mathrm{H} &, \quad
      \boldsymbol{I}_p \cdot \dot{\boldsymbol{\Omega}_i} = \boldsymbol{N}_i^\mathrm{H} + \boldsymbol{N}_i^\mathrm{b.h.}
    \end{split}
    \label{Newton_Euler}
  \end{equation}
  \normalsize
  \vspace{-4truemm}

  \noindent
ここでスイマー$i$について,
$\boldsymbol{R}_i$は位置,
$\boldsymbol{V}_i$は速度,
$\boldsymbol{Q_i}$は回転行列,
$\boldsymbol{\Omega}_i$は角速度,
$\mathrm{skew} (\boldsymbol{\Omega_i})$は$\boldsymbol{\Omega}_i$の交代行列を示す.
$M_\mathrm{p}$と$\boldsymbol{I}_\mathrm{p}$はそれぞれスイマーの質量と慣性モーメントを表す.
$\boldsymbol{F}_i^\mathrm{H}$は流体から受ける力,
$\boldsymbol{N}_i^\mathrm{H}$は流体から受けるトルクである.
球形粒子の場合,$N^\mathrm{H}$は式\eqref{nh}で表される.
  \vspace{-3truemm}
  \begin{equation}
    \boldsymbol{N}^\mathrm{H} = 4 \pi \mu a^3 \dot{\gamma}
    \label{nh}
  \end{equation}
  \vspace{-6truemm}

  \noindent
ここで,$\mu$は流体の粘度,$a$は球形粒子の半径,$\dot{\gamma}$はせん断速度である.
$\boldsymbol{N}_i^\mathrm{b.h.}$はbottom heavy性により生じるトルクで,式\eqref{nbh}で計算される\cite{3}.
  \vspace{-3truemm}
  \begin{equation}
    \boldsymbol{N}^\mathrm{b.h.} = \frac{4}{3} \pi a^3 \rho h \boldsymbol{e} \times \boldsymbol{g}
    \label{nbh}
  \end{equation}
  \vspace{-4truemm}

  \noindent
ここで,$a$はスイマーの半径,$\rho$は流体の密度,$h$は球の中心と重心との距離,$\boldsymbol{e}$はスイマーの方向ベクトル,$\boldsymbol{g}$は重力である.

  %% end 2.
  %% start 3.結果と考察

  \vspace{1.0truemm}
  \noindent
  {\bf \large 3.結果と考察}
  \par
直径$5\Delta$の単体スイマーについて,$64\Delta \times 128\Delta \times 64\Delta$の直方体システムでシミュレーションを行った.
このシステムでは,y軸方向に重力がかかり,x軸方向に流体の流れがある.
ここで,$\Delta$は格子間隔である.
通常の球形粒子とPuller型/Pusher型の3種について,
加えたせん断速度$\dot{\gamma}$の大きさと粒子の定常進行方向または,定常回転運動の有無を表したものがFig.\ref{snapshots}である.
  \vspace{-3truemm}
  \begin{figure}[h]
    \hspace{-3truemm}
    \centering
    \includegraphics[width=80truemm]{./images/rotation.png}
    \vspace{-6truemm}
    \hspace{-2truemm}
    \caption{Simulation snapshots}
    \label{snapshots}
  \end{figure}
  \vspace{-3truemm}

  \noindent
直線の矢印は定常せん断下での粒子の定常進行方向を表し,曲がった矢印は粒子が定常回転していることを表す.
粒子にかかるトルクのうちbottom heavy性に起因するトルク$N^\mathrm{b.h.}$が支配的な時,粒子の進行方向がある角度で固定され,
$N^\mathrm{H}$が支配的な時,粒子は回転すると考えられるが,その傾向が読み取れる.
今回のシミューションで得られた結果と理論的解析結果とがよく一致していることが確認できた.

  %% end 3.
  %% start 4.結言

  \vspace{1.0truemm}
  \noindent
  \textbf{\large 4.結言および今後の展望}
  \par
マイクロスイマーのモデルにsquirmerモデルを採用し,
スイマーにbottom heavy性を付与した上で単体粒子の挙動を解析した.
単体粒子の挙動については,理論的考察とよく一致することが確認でした.
マイクロスイマー分散系のレオロジー特性に対して解析を行うことが今後の課題である.
  \vspace{-5truemm}

  %% end 4.
  %% start 参考文献

  \renewcommand{\refname}{\normalsize 参考文献\vspace{-3truemm}}
  \begin{thebibliography}{9}
  %
  \bibitem{1}
    N. Oyama,J. J. Molina, and R. Yamamoto,\textit{Phys. Rev. E}, \textbf{112}, 1389-1397 (2017).\\
  %
  \vspace{-7truemm}
  %
  \bibitem{2}
    Y. Nakayam and R. Yamamoto, \textit{Phys. Rev. E}, \textbf{71},036707 (2005).\\
  %
  \vspace{-7truemm}
  %
  \bibitem{3}
    T. Ishikawa and T. J. Redley, J. \textit{Fluid Mech.}, \textbf{588} (2007).
  \end{thebibliography}

  %% end 参考文献
  %% satrt 締め

  % \vspace{-3truemm}
  \centering
  \underline{指導教員名\hspace{10truemm} 山本 量一 \hspace{20truemm} 印}

  %% end 締め

\end{document}

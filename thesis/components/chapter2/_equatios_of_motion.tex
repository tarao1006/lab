\subsubsection{運動方程式}
    \begin{align}
        \dot{\boldsymbol{R}_i} &= \boldsymbol{V}_i \\
        M_\mathrm{p} \dot{\boldsymbol{V}_i} &= \boldsymbol{F}^\mathrm{H}_i \\
        \boldsymbol{I}_\mathrm{p} \cdot \dot{\boldsymbol{\Omega}_i} &=
            \boldsymbol{N}^\mathrm{H}_i + \boldsymbol{N}^\mathrm{b.h.}_i
    \end{align}

ここで,$\boldsymbol{F}^\mathrm{H}_i$は流体から受ける力,
$\boldsymbol{N}^\mathrm{H}_i$は流体から受けるトルク,
$\boldsymbol{N}^\mathrm{b.h.}_i$はbottom heavy性によるトルクである.
$M_\mathrm{p}$は粒子の質量,
$\boldsymbol{I}_i$は慣性モーメント,
$\dot{\boldsymbol{R}_i}$は粒子の位置,
$\boldsymbol{V}_i$は粒子の速度,
$\dot{\boldsymbol{\Omega}_i}$は粒子の角速度である.
    \begin{figure}[htbp]
        % \includegraphics[]{}
        \label{bottom_heavyt_fig}
        \caption{bottom heavy性を有するsquirmer}
    \end{figure}

\noindent
Fig.\ref{bottom_heavy_fig}のように表されるsquirmerについて,
bottom heavy性によるトルクは式\eqref{bottom_heavy}で計算される.
    \begin{equation}
        \boldsymbol{N}^\mathrm{b.h.} = \frac{4}{3} \pi a^3 \rho h \boldsymbol{e} \times \boldsymbol{g}
        \label{bottom_heavy}
    \end{equation}

\noindent
ここで,$a$は粒子の半径,
$\rho$は粒子の密度,
$h$は球の中心と粒子の重心との距離,
$\boldsymbol{e}$は粒子の方向ベクトル,
$\boldsymbol{g}$は重力ベクトルである.

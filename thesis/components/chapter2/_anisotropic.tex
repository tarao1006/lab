\subsubsection{異方性の導入}
本研究では,bottom heavy性をsquirmerに付加することで,squirmerの進行方向に異方性を与えた.
その原理について述べる.
周辺流体に流れを生じさせる,Pusher型/Puller型のsquirmerの進行方向が鉛直上上向きからずれた場合,
Fig.\ref{anisotropy}のように擬似的なせん断流を考えることができる.
    \begin{figure}[htbp]
        % \includegraphics{}
        \label{anisotropy}
        \caption{suirmerの存在による応力}
    \end{figure}

\noindent
粒子が存在することによって生じる応力は式\eqref{particle_stresslet}のように表される\cite{}.
    \begin{equation}
        \boldsymbol{\Sigma}^\mathrm{p} = \frac{1}{V} \sum \boldsymbol{S}
        \label{particle_stresslet}
    \end{equation}

\noindent
ここで,$V$は系の体積,$\boldsymbol{S}$は単体粒子による応力である.
系に存在するのが単体のsquirmerのみの場合,その応力は式\eqref{solitary_squirmer}により計算される\cite{}.
    \begin{equation}
        \boldsymbol{S} = \\
            \frac{4}{3} \pi a^2 (3 \boldsymbol{e} \boldsymbol{e} - \boldsymbol{I}) B_2
        \label{solitary_squirmer}
    \end{equation}

\noindent
ここで,$a$はsquirmerの半径,
$\boldsymbol{e}$はsquirmerの方向ベクトル,
$\boldsymbol{I}$は単位行列である.
したがって,これらの2式から,squirmer単体が存在することで生じる応力の$xy$成分は,
    \begin{equation}
        \boldsymbol{\Sigma}^\mathrm{p}_{xy} = \
            \frac{4 \pi a^2 B_2}{V} e_x e_y
    \end{equation}
と表される.ここで,$e_i$はsquirmerの方向ベクトルの$i$成分である.
この応力は,粒子の進行方向により,Fig.\ref{graph_particle_stresslet}のように表される.
グラフから,粒子が定常的に回転している場合,この応力の影響はプラスとマイナスで打ち消されると考えられる.
    \begin{figure}[htbp]
        % \includegraphics[]{}
        \label{graph_particle_stresslet}
        \caption{$\boldsymbol{\Sigma}^\mathrm{p}_{xy}$と粒子の進行方向の関係}
    \end{figure}

\noindent
ここで,bottom heavy性によるトルクの影響を考える.
流体から受けるトルクとbottom heavy性によるトルクの大小関係から,粒子の回転が制限され,
粒子の進行方向がいずれかの進行方向に固定された結果,応力に影響を与えると考えられる.
例えば,$B_2$の値が正であるPuller型の場合,その進行方向が$0 〜 \pi / 2$の間に固定された場合,応力を大きくする方向に影響を与えると考えられる.
    \begin{figure}[htbp]
        % \includegraphics[]{}
        \label{graph_particle_stresslet2}
        \caption{粒子の進行方向が応力の大きさに与える影響}
    \end{figure}

\noindent
本研究では,$xy$平面上を$x$軸方向に流体の流れを生じさせ,
$y$軸方向に重力をかけた.

\subsection{Suirmerモデル}
\par
マイクロスイマーのモデルとして,Squirmerモデル\cite{}を採用した.
このモデルでは,球形の粒子表面において,粒子と流体の速度差が式\eqref{squirmer}で表される様なsliding境界条件を用いる.
    \begin{equation}
        \boldsymbol{u}^\mathrm{s} = 
            \sum_{n=1}^\infty\frac{2}{n(n + 1)} B_n P_n^\prime(\cos{\theta}) \sin{\theta} \hat{\boldsymbol{\theta}}
        \label{squirmer}
    \end{equation}

\noindent
ここで,$\boldsymbol{\hat{r}}$は粒子の中心からその表面上の点に向かう単位ベクトルであり,
$\theta$は$\boldsymbol{\hat{r}}$と粒子の方向ベクトル$\boldsymbol{\hat{e}}$とが成す角,
$\boldsymbol{\hat{\theta}}$は単位極角ベクトルである.
また,$\boldsymbol{u}^\mathrm{s}$はマイクロスイマー表面の重心に対するslide速度,
$B_n$は係数,
$P^\prime_n$は $n$次Legendre多項式の導関数である.
しかし,$n=1$の項はsquirmerの泳動速度を,$n=2$の項はsquirmerの存在によって生じる応力を決めるが,
$n \geq 3$の項は,泳動速度,および応力に影響を与えないので,式\eqref{squirmer}は,式\eqref{squirmer2}の様に簡略化できる.
    \begin{equation}
            \boldsymbol{u}^s =
                B_1\left(\sin{\theta} + \frac{\alpha}{2}\sin{2\theta}\right)\hat{\boldsymbol{\theta}}
        \label{squirmer2}
    \end{equation}
ここで,$B_1$はマイクロスイマーの進行速度の大きさ$(U = 2/3 B_1)$を与え,$\alpha = B_2/B_1$はその符号によりスイマーの種類を表す定数となる.
$\alpha > 0$をPuller型,$\alpha = 0$をNeutral型,$\alpha < 0$をPusher型と呼ぶ.
Puller型は,周辺流体に収縮流を生成し,Pusher型は伸長流を生成する.
    \begin{figure}[htbp]
        % \includegraphics[]{}
        \label{squirmer_flow}
        \caption{SquirmerモデルにおけるPusher型,Neutral型,Puller型の概略図}
    \end{figure}

\subsection{Smoothed Profile Method}
固液二相シミュレーションにおいて,固体-液体の移動境界の取り扱いは非常に重要である.
今回のシミュレーションでは,物理量とは異なる識別関数を導入し,仮想流体領域を考えるSmoothed Profile Method(SPM)\cite{}を用いた.
SPMでは,流体と粒子の界面にFig.\ref{spm_function}のような界面関数$\phi$を導入する.
この界面関数は,流体領域で$\phi=0$,粒子領域で$\phi=1$をとり,幅$\xi$の界面領域では滑らかに変化する連続関数である.
界面関数の導入により,境界条件を解く必要がなくなり,
以下で説明するNavier-Stokes方程式,および運動方程式を直接数値計算によって解くことが可能となる.
    \begin{figure}[htbp]
        % \includegraphics[]{}
        \label{spm_function}
        \caption{SPMの界面関数}
    \end{figure}


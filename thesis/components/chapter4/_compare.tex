\subsection{理論値との比較}
通常の球形粒子にbotton heavy性を仮定した場合について理論値との比較を行う.
この場合,\ref{sec:squirmer_model}で述べたように,粒子にかかるトルクは,式\eqref{eq:hydro_torque}のように表される.
    \begin{equation}
        \boldsymbol{N}^\mathrm{H} + \boldsymbol{N}^\mathrm{b.h.} = \
        4 \pi \mu a^3 \dot{\gamma} + \frac{4}{3} \pi a^3 \rho \boldsymbol{e} \times \boldsymbol{g}
        \label{eq:hydro_torque}
    \end{equation}

\noindent
したがって,各トルクとその和はFig.\ref{fig:sum_of_torque}のように表される.

    \begin{figure}[htbp]
        \centering
        \includegraphics[scale=1.0]{/Users/taiga/Projects/lab/thesis/components/chapter2/figs/sample.png}
        \caption{流体から受けるトルクとbottom heavy性によるトルクの和}
        \label{fig:sum_of_torque}
    \end{figure}

\noindent
パラメータを表\ref{tab:parameters}のように設定した場合,
$\dot{\gamma} < 0.05$の場合には粒子の進行方向は$0 < \theta < \pi / 2$のある角度に固定され,
$\dot{\gamma} = 0.05$の場合に,粒子の進行方向は$\theta = \pi /2$に固定され,
$\dot{\gamma} > 0.05$の場合に定常的に回転すると予想される.
ここで,$\Delta$は格子間距離であり,長さの単位として用いている.
また,$\mu$と$\rho$を基本単位として用いて,時間と重さの単位として用いている.

\begin{table}[htbp]
    \centering
    \caption{設定したパラメータ}
    \label{tab:parameters}
    \begin{tabular}{cc}
        \hline
        $\mu$ & $1.0$ \\
        $a$ & $5.0 \Delta$ \\
        $\rho$ & $1.0$ \\
        $h$ & $2.5\Delta$ \\
        $\boldsymbol{g}$ & (0, 0.06, 0) \\
        \hline
    \end{tabular}
\end{table}

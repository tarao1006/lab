\subsection{シミュレーション結果}
Fig.\ref{fig:result}はシミュレーション結果を模式的に表したものである.

    \begin{figure}[htbp]
        \centering
        \includegraphics[scale=1.0]{/Users/taiga/Projects/lab/thesis/components/chapter2/figs/sample.png}
        \caption{シミュレーションの模式図}
        \label{fig:result}
    \end{figure}

\noindent
上段は,通常の球形粒子にbottom heavy性を仮定したもの,
中段は,Puller型のsquirmer,
下段は,Pusher型のsquirmerのシミュレーション結果である.
直線の矢印は定常せん断下での粒子の定常進行方向を表し,曲がった矢印は粒子が定常回転していることを表す.
図の右側に行くにつれてせん断速度が大きいシミュレーションを表す.
また,せん断は図の右向きに,重力は図の下向きにかかっている.
この図より,せん断速度が小さい場合には,粒子はある進行方向に固定され,定常的にその方向に進むのに対し,
せん断速度が大きい場合には,定常的な回転運動を始めることが分かる.
これは,トルクと粒子の回転\ref{sec:rotation}で述べた予想に反しないことが分かる.

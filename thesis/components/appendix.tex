\documentclass[12pt, a4j, dvipdfmx]{jarticle}
\usepackage{amsmath}
\usepackage{setspace}
\usepackage{mathtools}
\usepackage[dvipdfmx]{graphicx}  % Include figure files
% \usepackage[varg]{txfonts}
\usepackage{bm}  % bold math
\usepackage{here}
\usepackage{array}
\usepackage[T1]{fontenc}
\usepackage{etoolbox}
\usepackage[top=30truemm,bottom=30truemm,left=25truemm,right=25truemm]{geometry}  % 余白
\usepackage{comment}
\usepackage{cases}
\usepackage[version=3]{mhchem}
\usepackage{layout}
\usepackage{wrapfig}
\usepackage{indentfirst}
\usepackage{txfonts}


%\renewcommand{\indent}{\hspace*{1zw}}
\renewcommand{\abstractname}{}
\renewcommand{\figurename}{Fig.}
\renewcommand{\thefootnote}{\fnsymbol{footnote}}
\renewcommand{\thesection}{\arabic{section}}


\pagestyle{plain}
%%%%%%%%%%%%%%%%%%%%%%%%%%%%%%%%%%%%%%%%%%%%%%%%%%%%%%%%%%%%%%%%%%%%%%%%%%
\makeatletter%% プリアンブルで定義する場合は必須
\patchcmd{\maketitle}{\@fnsymbol}{\@alph}{}{}  % Footnote numbers from symbols to small letters
\long\def\@makecaption#1#2{% \@makecaption を再定義します
  \vskip\abovecaptionskip
  \iftdir\sbox\@tempboxa{#1\hskip1zw#2}%
    \else\sbox\@tempboxa{#1~ #2}  % ここの : を ~ に変更する
  \fi
  \ifdim \wd\@tempboxa >\hsize% 
    \iftdir #1\hskip1zw#2\relax\par
      \else #1~ #2\relax\par\fi  % ここの : を ~ に変更する
  \else
    \global \@minipagefalse
    \hbox to\hsize{\hfil\box\@tempboxa\hfil}  % センタリング
%   \hbox to\hsize{\box\@tempboxa\hfil}%      左詰め
%   \hbox to\hsize{\hfil\box\@tempboxa}%      右詰め
  \fi
  \vskip\belowcaptionskip}
 \DeclareRobustCommand\cite{\unskip
\@ifnextchar[{\@tempswatrue\@citex}{\@tempswafalse\@citex[]}}
 \def\@cite#1#2{$^{[\hbox{\scriptsize{#1\if@tempswa , #2\fi}]}}$}
 \def\@biblabel#1{[#1]}
\makeatother%% プリアンブルで定義する場合は必須

\setlength{\columnsep}{8  truemm}
\setlength{\linewidth}{90 truemm}


\begin{document}


\addcontentsline{toc}{section}{Appendix}
\section*{Appendix}
\setcounter{subsection}{0}
\def\thesubsection{\Alph{subsection}}
\subsection{パラメーター一覧}

\begin{table}[H]
  \centering
  \begin{tabular}{lcr}
    \hline
    記号  & パラメーター \\
    \hline \hline
    $\Delta$ &格子幅\\
    $a$  & 粒子半径\\
    $d$  & 粒子径\\
    $\lambda_B$  & ビエルム長\\
    $e$ &  電気素量\\
    $k_\textrm{B}$  & ボルツマン定数\\
    $T$ &絶対温度\\
    $E$  & 電場強度\\
    $\epsilon$  & 誘電率\\
    $\eta$  & 溶媒の粘度\\
    $\rho$  & 溶媒の密度\\
    $D_\alpha$ &  拡散係数\\
    $z_\alpha$ &  $\alpha$種のイオン価数\\
    $\Psi$  & 静電ポテンシャル\\
    $\kappa^{-1}$  & デバイ長さ\\
    $\phi$  & 界面関数\\
    $\xi$  & 界面幅\\
    $\boldsymbol{f_{\textrm{p}}}$  & 粒子の剛体性を保証する力\\
    $\boldsymbol{n}$ & 単位ベクトル\\
    $\boldsymbol{I}$ &  単位テンソル\\
    $C_\alpha$  & イオン密度\\
    $C_\alpha^*$  & 補助イオン密度\\
    $\mu_\alpha$  & 化学ポテンシャル\\
    $M_{\textrm{p}}$ &  粒子質量\\
    $\boldsymbol{V_i}$  & 粒子$i$の速度\\
    $\boldsymbol{F_i^{\textrm{H}}}$ &  粒子$i$に働く流体からの力\\
    $\boldsymbol{F_i^{\textrm{C}}}$  & 粒子$i$に働く相互作用力\\
    $\boldsymbol{R_i}$  & 粒子$i$の位置\\
    $\boldsymbol{I_{\textrm{p}}}$ &  粒子の慣性モーメント\\
    $\boldsymbol{\Omega_i}$ &  粒子$i$の角速度\\
    $\boldsymbol{N_i^{\textrm{H}}}$  & 粒子$i$のトルク\\
    \hline
  \end{tabular}
\end{table}


\end{document}

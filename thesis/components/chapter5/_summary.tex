\subsection{まとめ}
本研究では,球状のsquirmerにおいて,
その重心が球の中心から後方にずれているbottom heavy性をシミュレーション上で新たに付加した後に,
bottom hevey性を付加した単体squirmerをジグザグ流を用いてせん断流を再現した系中に配置し,
その挙動を解析するシミュレーションを行った.
この性質を付加することにより,squirmerの進行方向,squirmerの重心のずれ具合,
および系にかかる重力の大きさから求まるトルクを新たに考慮することが可能となった.
Bottom heavy性を有するsquirmerの特徴的な挙動を調べるにあたり,
squirmerの存在による,系の応力および有効粘度への影響に注目した.
単体のsquirmerをせん断流下に配置すると,
せん断流の大きさに比例するトルクを受ける.
また,bottom heavy性によるトルクの大きさはsquirmerの進行方向に依存するサイン関数となる.
そのため,bottom heavy性によるトルクの最大の絶対値と流体から受けるトルクの大きさが同じ値となるせん断速度$\dot{\gamma}_c$が存在し,
その値との大小関係からsquirmerの回転運動の有無を予想できる.
せん断速度が$\dot{\gamma}_c$より小さい場合には,
bottom heavy性によるトルクが支配的となるため,
squirmerの進行方向は$0 \leq \theta < 2 \pi$の範囲内に固定され,回転運動をしないことが予想される.
また,せん断速度が$\dot{\gamma}_c$に等しい場合には,
bottom heavy性によるトルクが最小値をとる$\theta = \pi / 2$に固定され,
回転運動をしないことが予想される.
さらに,せん断速度が$\dot{\gamma}_c$より大きい場合には,
流体から受けるトルクが支配的となり,
squirmerは定常的に回転運動をすることが予想される.
Squirmerが存在することで生じる応力は,
squirmerの進行方向に依存した,サイン関数とコサイン関数を掛け合わせた関数となる.
そのため,squirmerの進行方向が固定される場合には,
squirmerの存在による応力が生じ,系の応力に影響を与えることが予想される.
一方,squirmerが定常的に等角速度で回転している場合には,
squirmerの存在による応力の時間平均をとるとプラスとマイナスが打ち消し合い,
系の応力には影響を与えないことが予想される.
さらに,有効粘度は,系の応力をせん断速度で割ることで求めることができるため,
系の応力への影響と同じ傾向が有効粘度の値にも見られることが予想される.
シミュレーションを行った結果,
球形粒子にbottom heavy性を仮定したものについて,
$\dot{\gamma}_c$の値,
およびせん断速度が$\dot{\gamma}_c$よりも小さい場合に固定される,squirmerの進行方向の角度が
シミュレーション結果と理論的に求まる値とで十分に一致していることを確認した.
さらに,せん断速度が$\dot{\gamma}_c$より小さい領域で,
Puller型の場合は,有効粘度を大きくする方向に,
Pusher型の場合は,有効粘度を小さくする方向にはたらくことが確認できた.
また,せん断速度が$\dot{\gamma}_c$より大きい領域では,
squirmerの種類によらず,ほぼ等しい有効粘度を示すことが確認できた.
この傾向は,Puller型であるクラミドモナス
およびPusher型であるE. Coli.を用いた実験により得られた結果と定性的に一致していることを確認した.

\subsection{まとめ}
本研究では,squirmerのbottom heavy性に起因するトルクを新たに付加できるように拡張を行った.
その結果,squirmerがシミュレーション開始時の進行方向に関わらず,
鉛直上向きを向こうとする挙動を再現することができた.
その後,bottom heavy性を付加したsquirmerをせん断流下に置くことで,
bottom heavy性によるトルクが支配的な場合には,定常的にある方向に進行し,
流体から受けるトルクが支配的な場合は,定常的に回転するという2種類の挙動を見せることを確認した.
また,その挙動が変化する閾値となるせん断速度の値が理論的な値とシミュレーション結果とで
よく一致していることを確認した.
さらに,せん断速度が小さい領域では,
Puller型は系の有効粘度を大きくする方向に,
Pusher型は系の有効粘度を小さくする方向にはたらくことを確認した.
また,せん断速度が大きい領域では,
squirmerの種類によらず,ほぼ等しい有効粘度を示すことを確認した.
加えて,それらの結果が実験結果と定性的に一致していることを確認した.

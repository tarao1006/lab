\documentclass[11pt, a4j, dvipdfmx]{jarticle}
\usepackage{amsmath}
\usepackage{setspace}
\usepackage{mathtools}
\usepackage[dvipdfmx]{graphicx}  % Include figure files
% \usepackage[varg]{txfonts}
\usepackage{bm}  % bold math
\usepackage{here}
\usepackage{array}
\usepackage[T1]{fontenc}
\usepackage{etoolbox}
\usepackage[top=30truemm,bottom=30truemm,left=25truemm,right=25truemm]{geometry}  % 余白
\usepackage{comment}
\usepackage{cases}
\usepackage[version=3]{mhchem}
\usepackage{layout}
\usepackage{wrapfig}
\usepackage{indentfirst}
\usepackage{txfonts}


%\renewcommand{\indent}{\hspace*{1zw}}
\renewcommand{\abstractname}{}
\renewcommand{\figurename}{Fig.}
\renewcommand{\thefootnote}{\fnsymbol{footnote}}
\renewcommand{\thesection}{\arabic{section}}


\pagestyle{plain}
%%%%%%%%%%%%%%%%%%%%%%%%%%%%%%%%%%%%%%%%%%%%%%%%%%%%%%%%%%%%%%%%%%%%%%%%%%
\makeatletter%% プリアンブルで定義する場合は必須
\patchcmd{\maketitle}{\@fnsymbol}{\@alph}{}{}  % Footnote numbers from symbols to small letters
\long\def\@makecaption#1#2{% \@makecaption を再定義します
  \vskip\abovecaptionskip
  \iftdir\sbox\@tempboxa{#1\hskip1zw#2}%
    \else\sbox\@tempboxa{#1~ #2}  % ここの : を ~ に変更する
  \fi
  \ifdim \wd\@tempboxa >\hsize% 
    \iftdir #1\hskip1zw#2\relax\par
      \else #1~ #2\relax\par\fi  % ここの : を ~ に変更する
  \else
    \global \@minipagefalse
    \hbox to\hsize{\hfil\box\@tempboxa\hfil}  % センタリング
%   \hbox to\hsize{\box\@tempboxa\hfil}%      左詰め
%   \hbox to\hsize{\hfil\box\@tempboxa}%      右詰め
  \fi
  \vskip\belowcaptionskip}
 \DeclareRobustCommand\cite{\unskip
\@ifnextchar[{\@tempswatrue\@citex}{\@tempswafalse\@citex[]}}
 \def\@cite#1#2{$^{[\hbox{\scriptsize{#1\if@tempswa , #2\fi}]}}$}
 \def\@biblabel#1{[#1]}
\makeatother%% プリアンブルで定義する場合は必須

\setlength{\columnsep}{8  truemm}
\setlength{\linewidth}{90 truemm}



\begin{document}


\addcontentsline{toc}{section}{参考文献}
\renewcommand{\refname}{参考文献}

\begin{thebibliography}{99}
%
\bibitem{Chlamydomonas}
K. Y. Wan and R. E. Goldstein, Coordinated beating of algal flagella is mediated by basal coupling. \textit{PANS}, \textbf{113}, 2784(2016)\\

\bibitem{effective_viscosity}
S. Rafa\"i, L. Jibuti, and P. Peyla, Effective Viscosity of Microswimmer Suspensions. \textit{Phys. Rev. E}, \textbf{104}, 098102(2010) \\

%
\bibitem{squirmer}
J. R. Blake, A spherical envelope approach to ciliary propulsion. \textit{J. Fluid Mech.}, \textbf{46}, 199(1971) \\

%
\bibitem{spm}
Y. Nakayama and R. Yamamoto, Simulation method to resolve hydrodynamic interactions in colloidal dispersions. \textit{Phys. Rev. E}, \textbf{71}, 036707(2005)\\

%
\bibitem{dilute_squirmer}
T. Ishikawa and T. J. Pedley, The rheology of a semi-dilute suspension of swimming model micro-organisms. \textit{J. Fluid Mech.}, \textbf{588}, 399(2007) \\

%
\bibitem{hidro_torque}
P. Hahn, A. Lamprecht and J. Dual, Numerical simulation of micro-particle rotation by the acoustic viscous torque. \textit{Lab Chip}, \textbf{16}, 4581(2016). \\

%
\bibitem{e_coli_experiment}
V. A. Martines, \textit{et al.}, A combined rheometry and imaging study of viscosity reduction in bacterial suspensions. \textit{PANS}, \textbf{117}, 2326(2020). 
\vspace{-8truemm}\\
\end{thebibliography}


\end{document}

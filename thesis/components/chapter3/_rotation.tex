\subsection{トルクと粒子の回転}
\label{sec:rotation}
$xy$平面上を$x$軸方向に流れるせん断流下に球形粒子が存在する場合,
流体から受けるトルクは式\eqref{eq:hydro_torque}で表される\cite{}.

    \begin{equation}
        N^\mathrm{H}_z = 4 \pi \mu a^3 \dot{\gamma}
        \label{eq:hydro_torque}
    \end{equation}

\noindent
ここで,$\mu$は流体の粘度,
$a$は粒子の半径,
$\dot{\gamma}$はせん断速度である.
流体から受けるトルクに加えて,
\ref{sec:equation_of_motion}で述べたbottom heavy性によるトルクを考えることで,
粒子の回転の有無を考えることができる.
本研究では,$xy$平面上を$x$軸方向に流れるせん断流を考えたため,
squirmerは$xy$平面上を移動すると考えた.
したがって,squirmerの方向ベクトルは$\theta$を用いて
$\boldsymbol{\hat{e}} = (\sin{\theta}, \cos{\theta}, 0)$
のように表すことができる.
また,系にかかる重力を$y$軸方向下向きとし,
$\boldsymbol{g} = (0, -g, 0)$と表すと,式\eqref{eq:bottom_heavy_torque}は,
式\eqref{eq:bottom_heavy_torque_elements}のように表される.

    \begin{align}
        \boldsymbol{N}^\mathrm{b.h.} &= \frac{4}{3} \pi \rho h \cdot (\sin{\theta}, \cos{\theta}, 0) \times (0, -g, 0) \notag \\
        \therefore N^\mathrm{b.h.}_z &= - \frac{4}{3} \pi \rho h g \sin{\theta}
        \label{eq:bottom_heavy_torque_elements}
    \end{align}

\noindent
これにより,流体から受けるトルクと,bottom heavy性によるトルクの和の$z$成分は,式\eqref{eq:sum_of_torque_z}のように表される.

    \begin{align}
        N_z &= N^\mathrm{H}_z + N^\mathrm{b.h.}_z \notag \\
            &= 4 \pi \mu a^3 \dot{\gamma} - \frac{4}{3} \pi \rho h g \sin{\theta}
        \label{eq:sum_of_torque_z}
    \end{align}

\noindent
したがって,粒子の進行方向と粒子にはたらくトルクの$z$成分の関係は,
せん断速度の大きさにより,Fig.\ref{fig:sum_torque}(a)〜(c)のように3種類のグラフで表すことができる.
せん断速度が小さい場合には,bottom heavy性によるトルクが支配的となり,Fig.\ref{fig:sum_torque}(a)のように,
トルクの和が0となる$0 \leq \theta < \pi / 2$の間の角度で粒子の進行方向が固定され,粒子は回転しないことが予想される.
また,Fig.\ref{fig:sum_torque}(b)のように,
流体から受けるトルクの値が,bottom heavy性によるトルクの大きさの最大値となる場合は,
粒子の進行方向は$\theta = \pi / 2$に固定され,粒子は回転しないことが予想される.
せん断速度がその値よりも大きくなると,Fig.\ref{fig:sum_torque}(c)のように,
流体から受けるトルクが支配的となり,粒子が定常的に回転することが予想される.
このように,流体から受けるトルクとbottom heavy性によるトルクの釣り合いから,粒子の進行方向が異方的になることが予想される.

\begin{figure}[H]
    \centering
    \includegraphics[scale=0.85]{/Users/taiga/Projects/lab/thesis/components/chapter3/figs/sum_torque.pdf}
    \caption{せん断速度の大きさの違いによる粒子にはたらくトルクの分類}
    \label{fig:sum_torque}
\end{figure}

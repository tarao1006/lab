\subsection{シミュレーション結果}
Fig.\ref{snapshots}はシミュレーションのスナップショットである.
    \begin{figure}[htbp]
        % \includegraphics[]{}
        \label{snapshots}
        \caption{シミュレーションのスナップショット}
    \end{figure}

\noindent
上段は,通常の球形粒子にbottom heavy性を仮定したもの,
中段は,Puller型のsquirmer,
下段は,Pusher型のsquirmerのシミュレーション結果である.
直線の矢印は定常せん断下での粒子の定常進行方向を表し,曲がった矢印は粒子が定常回転していることを表す.
せん断速度が小さい場合には,粒子はある進行方向に固定され,
せん断速度が大きくなるにつれ粒子の進行方向は$\pi / 2$に近づき,
$\pi / 2$を超えると回転運動を始めることが分かる.
これは,せん断速度が小さい時には,$N^\mathrm{b.h.}$が支配的であり,
せん断速度が大きくなるにつれ$N^\mathrm{H}$の影響が大きくなるからであると考えられる.

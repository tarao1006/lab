\subsection{理論値との比較}
通常の球形粒子にbotton heavy性を仮定した場合について理論値との比較を行う.
球形粒子がせん断によって受けるトルクは式\eqref{spherical_particle_nh}のように表される\cite{}.
    \begin{equation}
        \boldsymbol{N}^\mathrm{H} = \
            4 \pi \mu a^3 \dot{\gamma}
        \label{spherical_particle_nh}
    \end{equation}

\noindent
ここで,$\mu$は流体の粘度,
$\dot{\gamma}$はせん断速度である.
したがって,球形粒子にかかるトルクの総和は,
    \begin{equation}
        \boldsymbol{N}^\mathrm{H} + \boldsymbol{N}^\mathrm{b.h.} = \
        4 \pi \mu a^3 \dot{\gamma} + \frac{4}{3} \pi a^3 \rho \boldsymbol{e} \times \boldsymbol{g}
    \end{equation}

\noindent
と表される.
各トルクとその和はFig.\ref{}のように表される.

    \begin{figure}
    %    \includegraphics[]{}
        \label{}
        \caption{}
    \end{figure}
